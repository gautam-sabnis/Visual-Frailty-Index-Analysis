\usepackage{hyperref}
\usepackage{amsmath,amssymb,amsfonts,amsthm,enumerate,natbib,color,ifthen}
\usepackage{xargs}
%\usepackage{calrsfs}
%\usepackage[notref,notcite]{showkeys}  %  comment out for final version
%\renewcommand*\showkeyslabelformat[1]{\fbox{\normalfont\scriptsize\sffamily#1}}   % for showkeys
\usepackage[textwidth=4cm, textsize=footnotesize]{todonotes}
\setlength{\marginparwidth}{4cm}    %  this goes with todonotes
\usepackage[latin1]{inputenc}



\usepackage{adjustbox}
\usepackage{multicol,booktabs,colortbl,tabularx}
\usepackage{caption,subcaption}
\usepackage{float}
%\usepackage[caption = false]{subfig}
\usepackage{graphicx}

\usepackage{aliascnt,bbm}
%\usepackage{algorithmicx}

%\usepackage{showlabels}
%\usepackage[latin1]{inputenc}
\renewcommand{\baselinestretch}{1.2}

%\topmargin -0.5cm
%\oddsidemargin -0.25cm
%\evensidemargin -0.25cm
%
%\textheight 9in
\textwidth 5.6in



% NOTATIONS
%%%%%%%%%%%%%
%%%%%%%%%%%%%
% Ensembles
\def\nset{{\mathbb{N}}}
\def\rset{\mathbb R}
\def\zset{\mathbb Z}
\def\qset{\mathbb Q}
\def\eqsp{\;}


\newcommand{\eg}{e.g.}
\newcommand{\ie}{i.e.}
\newcommand{\wrt}{with respect to}
\newcommand{\as}{\text{a.s.}}
\newcommand{\pscal}[2]{\left\langle#1,#2\right\rangle}

\newcommand{\un}{\ensuremath{\mathbbm{1}}}
\newcommand{\eqdef}{\ensuremath{\stackrel{\mathrm{def}}{=}}}
\newcommand{\eps}{\varepsilon}


\def\Eset{\mathsf{E}}
\def\Xset{\mathcal{Z}} % Espace d 'etat
\def\Zset{\mathcal{Z}} % Espace d 'etat
\def\Vset{\mathsf{V}} % Espace d 'etat
\def\Yset{\mathsf{Y}} % Espace d 'etat
\def\Xsigma{\mathcal{X}} % tribu sur X
\def\Tsigma{\mathcal{B}(\Theta)} % tribu sut Theta
\def\F{\mathcal{F}} % filtration
\def\B{\mathcal{B}} % filtration
\def\cB{\mathsf{B}} % filtration
\def\barB{\overline{B}} % filtration
\def\e{\mathcal{E}}
\def\dist{\textsf{d}}
\def\q{\mathsf{q}}
\def\E{\mathbb{E}}
\def\N{\mathcal{N}}
\def\M{\mathcal{M}}
\def\m{\mathsf{m}}
\def\Mt{\mathcal{M}_{2,\textsf{w}}^{(0)}}
\def\G{\mathcal{G}}
\def\D{\mathcal{D}}
\def\A{\mathcal{A}}
\def\bV{\mathcal{V}}
\def\cV{\check{\mathcal{V}}}
\def\H{\mathcal{H}}
\def\cov{\mathbb{C}\mbox{ov}}
\def\calK{\mathcal{K}}
\newcommandx\sequence[3][2=t,3=\zset]
{\ifthenelse{\equal{#3}{}}{\ensuremath{\{ #1_{#2}\}}}{\ensuremath{\{ #1_{#2}, \eqsp #2 \in #3 \}}}}

%\DeclareMathAlphabet{\pazocal}{OMS}{zplm}{m}{n}
\newcommand{\Sa}{\mathcal{S}}
%\newcommand{\Sb}{\pazocal{S}}




\def\PP{\mathbb{P}} % proba
\newcommand{\CPP}[3][]
{\ifthenelse{\equal{#1}{}}{{\mathbb P}\left(\left. #2 \, \right| #3 \right)}{{\mathbb P}_{#1}\left(\left. #2 \, \right | #3 \right)}}
\def\PE{\mathbb{E}} % esperance
\newcommand{\CPE}[3][]
{\ifthenelse{\equal{#1}{}}{{\mathbb E}\left[\left. #2 \, \right| #3 \right]}{{\mathbb E}_{#1}\left[\left. #2 \, \right | #3 \right]}}

\def\bPP{\overline{\mathbb{P}}} % proba sur espace double
\def\bPE{\overline{\mathbb{E}}} % esperance sur espace double
\def\cPP{\check{\mathbb{P}}} % proba sur espace quadruple
\def\cPE{\check{\mathbb{E}}} % esperance sur espace quadruple
\def\cP{\check{P}} % proba
\def\wtilde{\widetilde} % proba



\def\lyap{\omega}
\def\W{\mathcal{W}}
\def\K{\mathcal{K}}
\def\L{\mathcal{L}} % espace des fonctions
\def\Vs{V_\star}
\def\V{\mathbb{V}}
\def\ns{n_\star}
\def\tv{\mathrm{tv}}
\def\tvs{\mathrm{tv(s)}}
\def\tauout{\stackrel{\leftarrow}{\tau}} % temps de sortie
\def\compact{\mathsf{K}}  % compact
\def\Cset{\mathcal{C}} % Petite set
\def\Dset{\mathcal{D}} % Generic set
\newcommand{\dlim}{\ensuremath{\stackrel{\mathcal{D}}{\longrightarrow}}}
\newcommand{\plim}[1]{\ensuremath{\stackrel{\mathrm{P}_{#1}}{\longrightarrow}}}
\newcommand{\aslim}{\ensuremath{\stackrel{\text{a.s.}}{\longrightarrow}}}
\newcommand\Plim[1]{\mathbb{Q}_{#1}}
\newcommand{\abs}[1]{\left\vert#1\right\vert}
\newcommand{\norm}[1]{\left\Vert#1\right\Vert}
\newcommand{\normoff}[1]{\left\Vert#1\right\Vert_{1,\textsf{off}}}
\newcommand{\normfro}[1]{\left\Vert#1\right\Vert_{\textsf{F}}}

\newcommand{\tnorm}[1]{\left\vert\!\left\vert\!\left\vert#1\right\vert\!\right\vert\!\right\vert}
\newcommand{\nnorm}[1]{\left\Vert#1\right\Vert_2}
\newcommand{\inte}[1]{\stackrel{\circ}{#1}}


\def\Rg{\mathsf{Rg}}
\def\barlambda{\underline{\Lambda}}
\def\Mb{\mathcal{M}_\textsf{b}}
\def\b{\textsf{b}}
\def\piq{\textsf{q}}
\def\I{\textsf{I}}
\def\t{\textsf{T}}
\def\w{\textsf{w}}
\def\c{\textsf{c}}
\def\sp{\textsf{sp}}
\def\deg{\textsf{deg}}
\def\Prox{\operatorname{Prox}}
\def\Proxj{\textsf{Prox}^{[j]}}
\def \thetagl{\hat\theta_{\textsf{gl}}}
\def \thetalasso{\hat\theta_{\textsf{lasso}}}
\def\vec{\textsf{Vec}}
\def\mat{\textsf{Mat}}
\def\r{\textsf{r}}


\usepackage{bm}%bold matH
\def\bzero{{\bf 0}}







% DIALOGUES ENTRE AUTEURS
%%%%%%%%%%%%%%%%%%%%%%%%%
%\newcommand{\comment}[2]{\begin{quote}\color{#1}\begin{sffamily} #2 \end{sffamily}\color{black}\end{quote}}
%\newcommand{\note}[2]{\marginpar{\parbox[t]{\noteWidth}{\raggedright\textcolor{blue}{\tiny{#2}} \textcolor{red}{\tiny{#1}}}}}
%
%\def\yves{blue}
%\def\eric{red}
%\newlength{\noteWidth}
%\setlength{\noteWidth}{.6in}
%\newcommand{\yves}[1]{\todo[color=green!20]{{\bf Y:} #1}}
%\newcommand{\anwesha}[1]{\todo[color=blue!20]{{\bf A:} #1}}




\theoremstyle{plain}
\newtheorem{theorem}{Theorem}
\newtheorem{assumption}{H\hspace{-3pt}}
\newtheorem{assumptionB}{B\hspace{-3pt}}
\newtheorem{assumptionC}{C\hspace{-3pt}}
\newtheorem{problem}{P}


\newaliascnt{proposition}{theorem}
\newtheorem{proposition}[proposition]{Proposition}
\aliascntresetthe{proposition}

\newaliascnt{lemma}{theorem}
\newtheorem{lemma}[lemma]{Lemma}
\aliascntresetthe{lemma}
\newaliascnt{corollary}{theorem}
\newtheorem{corollary}[corollary]{Corollary}
\aliascntresetthe{corollary}


 \newtheorem{conjecture}{Conjecture}[section]
\newtheorem{hypothese}{Hypothesis}
\newtheorem{fact}[theorem]{Fact}


\theoremstyle{definition}
\newaliascnt{definition}{theorem}
\newtheorem{definition}[definition]{Definition}
\aliascntresetthe{definition}

%\newaliascnt{algorithm}{theorem}             % uncomment these 3 lines and comment line 90 if dont want float
%\newtheorem{algorithm}[algorithm]{Algorithm}
%\aliascntresetthe{algorithm}
\newtheorem{algorithm}{Algorithm}

\newaliascnt{remark}{theorem}
\newtheorem{remark}[remark]{Remark}
\aliascntresetthe{remark}

\newaliascnt{example}{theorem}
\newtheorem{example}[example]{Example}
\aliascntresetthe{example}



\providecommand*{\definitionautorefname}{Definition}
\providecommand*{\lemmaautorefname}{Lemma}
\renewcommand*{\subsectionautorefname}{Section}
\renewcommand*{\subsubsectionautorefname}{Section}
\providecommand*{\exampleautorefname}{Example}
\providecommand*{\propositionautorefname}{Proposition}
\providecommand*{\procautorefname}{Procedure}
\providecommand*{\exerciseautorefname}{Exercise}
\providecommand*{\corollaryautorefname}{Corollary}
\providecommand*{\remarkautorefname}{Remark}
\providecommand*{\algorithmautorefname}{Algorithm}

\providecommand*{\assumptionautorefname}{H\hspace*{-2.5pt}}  %hspace is stretchable, but \! is not and \!\! is too much
\providecommand*{\problemautorefname}{(P)\hspace*{-2.5pt}}  %hspace is stretchable, but \! is not and \!\! is too much


\newcommand{\coint}[1]{\left[#1\right)}
\newcommand{\ocint}[1]{\left(#1\right]}
\newcommand{\ooint}[1]{\left(#1\right)}
\newcommand{\ccint}[1]{\left[#1\right]}

\def\rmd{\mathrm{d}}
\def\rme{\mathrm{e}}
\def\PG{\operatorname{PG}}
\def\1{\mathbbm{1}}
\def\gauss{\operatorname{N}}
\def\bU{\mathbf{U}}
\def\bY{\mathbf{Y}}
\def\bu{\mathbf{u}}
\def\bW{\mathbf{W}}
\def\bw{\mathbf{w}}
\def\stdnor{\mathbf{N}}
\def\iid{i.i.d.}
\newcommand{\ball}[2]{\mathcal{B}(#1,#2)} 


%From Rahul
\newcommand{\sbt}{\mbox{subject to}}
\DeclareMathOperator*{\mini}{minimize}
\DeclareMathOperator*{\maxi}{maximize}
%\def\argmin{\operatorname{Argmin}}
\DeclareMathOperator*{\argmin}{{\textsf{Argmin}}}





